% Options for packages loaded elsewhere
\PassOptionsToPackage{unicode}{hyperref}
\PassOptionsToPackage{hyphens}{url}
%
\documentclass[
  ignorenonframetext,
]{beamer}
\usepackage{pgfpages}
\setbeamertemplate{caption}[numbered]
\setbeamertemplate{caption label separator}{: }
\setbeamercolor{caption name}{fg=normal text.fg}
\beamertemplatenavigationsymbolsempty
% Prevent slide breaks in the middle of a paragraph
\widowpenalties 1 10000
\raggedbottom
\setbeamertemplate{part page}{
  \centering
  \begin{beamercolorbox}[sep=16pt,center]{part title}
    \usebeamerfont{part title}\insertpart\par
  \end{beamercolorbox}
}
\setbeamertemplate{section page}{
  \centering
  \begin{beamercolorbox}[sep=12pt,center]{part title}
    \usebeamerfont{section title}\insertsection\par
  \end{beamercolorbox}
}
\setbeamertemplate{subsection page}{
  \centering
  \begin{beamercolorbox}[sep=8pt,center]{part title}
    \usebeamerfont{subsection title}\insertsubsection\par
  \end{beamercolorbox}
}
\AtBeginPart{
  \frame{\partpage}
}
\AtBeginSection{
  \ifbibliography
  \else
    \frame{\sectionpage}
  \fi
}
\AtBeginSubsection{
  \frame{\subsectionpage}
}
\usepackage{amsmath,amssymb}
\usepackage{lmodern}
\usepackage{iftex}
\ifPDFTeX
  \usepackage[T1]{fontenc}
  \usepackage[utf8]{inputenc}
  \usepackage{textcomp} % provide euro and other symbols
\else % if luatex or xetex
  \usepackage{unicode-math}
  \defaultfontfeatures{Scale=MatchLowercase}
  \defaultfontfeatures[\rmfamily]{Ligatures=TeX,Scale=1}
\fi
\usetheme[]{metropolis}
% Use upquote if available, for straight quotes in verbatim environments
\IfFileExists{upquote.sty}{\usepackage{upquote}}{}
\IfFileExists{microtype.sty}{% use microtype if available
  \usepackage[]{microtype}
  \UseMicrotypeSet[protrusion]{basicmath} % disable protrusion for tt fonts
}{}
\makeatletter
\@ifundefined{KOMAClassName}{% if non-KOMA class
  \IfFileExists{parskip.sty}{%
    \usepackage{parskip}
  }{% else
    \setlength{\parindent}{0pt}
    \setlength{\parskip}{6pt plus 2pt minus 1pt}}
}{% if KOMA class
  \KOMAoptions{parskip=half}}
\makeatother
\usepackage{xcolor}
\newif\ifbibliography
\usepackage{color}
\usepackage{fancyvrb}
\newcommand{\VerbBar}{|}
\newcommand{\VERB}{\Verb[commandchars=\\\{\}]}
\DefineVerbatimEnvironment{Highlighting}{Verbatim}{commandchars=\\\{\}}
% Add ',fontsize=\small' for more characters per line
\usepackage{framed}
\definecolor{shadecolor}{RGB}{248,248,248}
\newenvironment{Shaded}{\begin{snugshade}}{\end{snugshade}}
\newcommand{\AlertTok}[1]{\textcolor[rgb]{0.94,0.16,0.16}{#1}}
\newcommand{\AnnotationTok}[1]{\textcolor[rgb]{0.56,0.35,0.01}{\textbf{\textit{#1}}}}
\newcommand{\AttributeTok}[1]{\textcolor[rgb]{0.77,0.63,0.00}{#1}}
\newcommand{\BaseNTok}[1]{\textcolor[rgb]{0.00,0.00,0.81}{#1}}
\newcommand{\BuiltInTok}[1]{#1}
\newcommand{\CharTok}[1]{\textcolor[rgb]{0.31,0.60,0.02}{#1}}
\newcommand{\CommentTok}[1]{\textcolor[rgb]{0.56,0.35,0.01}{\textit{#1}}}
\newcommand{\CommentVarTok}[1]{\textcolor[rgb]{0.56,0.35,0.01}{\textbf{\textit{#1}}}}
\newcommand{\ConstantTok}[1]{\textcolor[rgb]{0.00,0.00,0.00}{#1}}
\newcommand{\ControlFlowTok}[1]{\textcolor[rgb]{0.13,0.29,0.53}{\textbf{#1}}}
\newcommand{\DataTypeTok}[1]{\textcolor[rgb]{0.13,0.29,0.53}{#1}}
\newcommand{\DecValTok}[1]{\textcolor[rgb]{0.00,0.00,0.81}{#1}}
\newcommand{\DocumentationTok}[1]{\textcolor[rgb]{0.56,0.35,0.01}{\textbf{\textit{#1}}}}
\newcommand{\ErrorTok}[1]{\textcolor[rgb]{0.64,0.00,0.00}{\textbf{#1}}}
\newcommand{\ExtensionTok}[1]{#1}
\newcommand{\FloatTok}[1]{\textcolor[rgb]{0.00,0.00,0.81}{#1}}
\newcommand{\FunctionTok}[1]{\textcolor[rgb]{0.00,0.00,0.00}{#1}}
\newcommand{\ImportTok}[1]{#1}
\newcommand{\InformationTok}[1]{\textcolor[rgb]{0.56,0.35,0.01}{\textbf{\textit{#1}}}}
\newcommand{\KeywordTok}[1]{\textcolor[rgb]{0.13,0.29,0.53}{\textbf{#1}}}
\newcommand{\NormalTok}[1]{#1}
\newcommand{\OperatorTok}[1]{\textcolor[rgb]{0.81,0.36,0.00}{\textbf{#1}}}
\newcommand{\OtherTok}[1]{\textcolor[rgb]{0.56,0.35,0.01}{#1}}
\newcommand{\PreprocessorTok}[1]{\textcolor[rgb]{0.56,0.35,0.01}{\textit{#1}}}
\newcommand{\RegionMarkerTok}[1]{#1}
\newcommand{\SpecialCharTok}[1]{\textcolor[rgb]{0.00,0.00,0.00}{#1}}
\newcommand{\SpecialStringTok}[1]{\textcolor[rgb]{0.31,0.60,0.02}{#1}}
\newcommand{\StringTok}[1]{\textcolor[rgb]{0.31,0.60,0.02}{#1}}
\newcommand{\VariableTok}[1]{\textcolor[rgb]{0.00,0.00,0.00}{#1}}
\newcommand{\VerbatimStringTok}[1]{\textcolor[rgb]{0.31,0.60,0.02}{#1}}
\newcommand{\WarningTok}[1]{\textcolor[rgb]{0.56,0.35,0.01}{\textbf{\textit{#1}}}}
\usepackage{longtable,booktabs,array}
\usepackage{calc} % for calculating minipage widths
\usepackage{caption}
% Make caption package work with longtable
\makeatletter
\def\fnum@table{\tablename~\thetable}
\makeatother
\usepackage{graphicx}
\makeatletter
\def\maxwidth{\ifdim\Gin@nat@width>\linewidth\linewidth\else\Gin@nat@width\fi}
\def\maxheight{\ifdim\Gin@nat@height>\textheight\textheight\else\Gin@nat@height\fi}
\makeatother
% Scale images if necessary, so that they will not overflow the page
% margins by default, and it is still possible to overwrite the defaults
% using explicit options in \includegraphics[width, height, ...]{}
\setkeys{Gin}{width=\maxwidth,height=\maxheight,keepaspectratio}
% Set default figure placement to htbp
\makeatletter
\def\fps@figure{htbp}
\makeatother
\setlength{\emergencystretch}{3em} % prevent overfull lines
\providecommand{\tightlist}{%
  \setlength{\itemsep}{0pt}\setlength{\parskip}{0pt}}
\setcounter{secnumdepth}{-\maxdimen} % remove section numbering
\setsansfont{Ubuntu}
\setmonofont{Ubuntu Mono}
\metroset{numbering=counter}
\usepackage[round]{natbib}
\usepackage{linguex}
\usepackage{qtree}
\usepackage{stmaryrd}
\usepackage{hyperref}
\hypersetup{ colorlinks=true, linkcolor=blue, filecolor=magenta, urlcolor=cyan, }
\urlstyle{same}
\newcommand{\cond}[1]{\textsc{#1}}
\ifLuaTeX
  \usepackage{selnolig}  % disable illegal ligatures
\fi
\usepackage[]{natbib}
\bibliographystyle{plainnat}
\IfFileExists{bookmark.sty}{\usepackage{bookmark}}{\usepackage{hyperref}}
\IfFileExists{xurl.sty}{\usepackage{xurl}}{} % add URL line breaks if available
\urlstyle{same} % disable monospaced font for URLs
\hypersetup{
  pdftitle={Equatives and two theories of negative concord},
  pdfauthor={Mojmír Dočekal},
  hidelinks,
  pdfcreator={LaTeX via pandoc}}

\title{Equatives and two theories of negative concord}
\subtitle{experimental evidence from Czech}
\author{Mojmír Dočekal}
\date{Berlin, 5-10-2022}
\institute{FDSL 15}

\begin{document}
\frame{\titlepage}

\begin{frame}[allowframebreaks]
  \tableofcontents[hideallsubsections]
\end{frame}
\begin{frame}{Intro}
\protect\hypertarget{intro}{}
\begin{itemize}
\tightlist
\item
  talk about expressions depending on the polarity
\item
  evidence: Czech (strict negative concord language)
\item
  data gathered from many experiments: long and extensive collaborative
  work with Jakub Dotlačil, Iveta Šafratová, Tereza Slunská, Martin
  Juřen and many other linguists in Brno and around
\item
  add to experimental research on NPIs:
  \cite{Chemla-Homer-Rothschild-NPI,gajewski2016another,alexandropoulou2020there}
  a.o.
\item
  more specifically: \cite{djarv2018cognitive,schwarz2020italian} and
  their experimental work on cross-linguistic variation in NPI licensing
  (following \citealt{chierchia2019factivity})
\end{itemize}
\end{frame}

\begin{frame}
\begin{itemize}
\tightlist
\item
  empirically, the talk is about Czech strong NPIs and neg-words
\item
  \emph{ani jeden} `even one' vs.~\textit{žádný} `no' (neg-word)
\item
  in the majority of contexts: interchangeable -- \Next
\end{itemize}

\exg. Petr nepotkal \{ani jednoho/žádného\} studenta.\\
Petr neg-met strong NPI/neg-word student\\
`Petr didn't meet even one/any student.'

\begin{itemize}
\tightlist
\item
  strong NPIs (theoretical framework: \citealt{gajewski2011licensing})
  but with the unlikelihood presupposition (English \emph{ANY}:
  \citealt{krifka1995semantics}, Hindi \emph{ek bhii}:
  \citealt{lahiri1998focus}, English \emph{even one}:
  \citealt{crnivc2014non})
\end{itemize}
\end{frame}

\begin{frame}
\begin{itemize}
\tightlist
\item
  \emph{ani}: the unlikelihood presupposition of English \emph{even} but
  limited to strong NPI contexts
\item
  strongest (unlikely) prejacent: entailing all the alternatives
\end{itemize}

\exg. FC Barcelona nedala \{ani jeden/\#ani deset\} gól/ů.\\
FC Barcelona neg-gave even one/\#even ten goal(s)\\
`FC Barcelona didn't score even one/\#ten goal(s).'

~
\end{frame}

\begin{frame}
Czech neg-words

\begin{itemize}
\tightlist
\item
  similar to Italian neg-words (\emph{niente}, e.g.:
  \citealt{ladusaw1992expressing}) but as in all Slavic languages
  (strict negative-concord: \citealt{zeijlstra2004sentential}) in
  majority of contexts require verbal negation (in the same clause)
\end{itemize}

\ex. \ag. Petr nedal žádný gól.\\
Petr neg-scored neg-word goal\\
`Petr didn't score any goal.' \bg. Nikdo \{nepřišel/\#přišel\}.\\
neg-word neg-came/came\\
`Nobody came.' \cg. *Petr neřekl, že nikdo přišel.\\
Petr neg-said that neg-word came\\
\hspace*{0.333em}

~
\end{frame}

\begin{frame}
\begin{itemize}
\tightlist
\item
  the most influential analysis of neg-words: syntactic approach
  (\citealt{zeijlstra2004sentential} a.o.)
\item
  in strict negative concord languages, all neg-words (and the verbal)
  negation carry {[}uNeg{]} and are checked against {[}iNeg{]} (covert)
  operator with the semantics of \(\neg\)
\item
  part of the talk: experimental support for an alternative, semantic
  theory of neg-words (\citealt{ovalle2004double,kuhn2022dynamics})
\end{itemize}
\end{frame}

\begin{frame}
\begin{itemize}
\tightlist
\item
  equatives: one of the contexts where strong NPIs and neg-words
  distribution diverge
\item
  Czech equatives don't license strong \& weak NPIs (like German and
  many other non-English NPIs: see \citealt{krifka1992some}) but license
  neg-words
\item
  surprising against English and standard theories of equatives
  \cite{stechow1984comparing,beck201913} a.o.
\item
  one of the environments where the contrast is most robust but still
  there's a variation: some speakers treat \emph{ani} as neg-word
\end{itemize}

\exg. Petr je tak vysoký jako \{\#ani jeden/žádný\} jiný student.\\
Petr is so tall how strong NPI/neg-word other student.\\
\hspace*{0.333em}

\ex. Paris is as quiet as ever.

~
\end{frame}

\begin{frame}
\begin{block}{Negative quantifiers, NPIs, neg-words and variation}
\protect\hypertarget{negative-quantifiers-npis-neg-words-and-variation}{}
\begin{itemize}
\tightlist
\item
  connected to the recent work on English NPIs vs.~negative quantifiers
  and its variation
\item
  \citealt{tottie1991negation,burnett2018structural}: NPIs replace
  negative quantifiers in some (lower, e.g.) syntactic domains

  \begin{itemize}
  \tightlist
  \item
    historical and social factors are real but weaker than grammatical
  \end{itemize}
\item
  similarly: \cite{burnett2015variable}: the variable negative concord
  in Québec French
\item
  experimental work: search for factors (grammatical and social as well)
\item
  plus explaining the puzzling equative pattern
\end{itemize}
\end{block}
\end{frame}

\begin{frame}
\begin{block}{The empirical and theoretical questions}
\protect\hypertarget{the-empirical-and-theoretical-questions}{}
\ex. Question 1: How to explain the unpredicted acceptability of Czech
neg-words in equatives (and NPIs unavailability)?\\
\a. Especially considering the monotonic properties of equatives.

\begin{itemize}
\tightlist
\item
  experimental data give us precise enough clues
\end{itemize}

\ex. Question2: \a. How can we explain microvariation by grammatical
(semantic) factors? \b. Is part of the variation caused by social
factors? \z.

~
\end{block}
\end{frame}

\begin{frame}{Experiment}
\protect\hypertarget{experiment}{}
\begin{itemize}
\tightlist
\item
  the experiment was run online on the L-Rex platform
\item
  mostly students of MUNI (Brno) and UK (Prague)
\item
  105 participants, 82 passed the fillers and were included in the stats
\item
  each questionnaire: 64 items, 48 randomized lists
\item
  3 demographic-related questions:

  \begin{itemize}
  \tightlist
  \item
    age
  \item
    region
  \item
    daily reading time (books, etc.)
  \end{itemize}
\end{itemize}
\end{frame}

\begin{frame}
Two parts of the experiment:

\begin{enumerate}
\item
  acceptability judgment task (no context)
\item
  acceptability judgment task against probability/scalarity manipulated
  context
\end{enumerate}

\begin{itemize}
\item
  both parts: participants judged the acceptability of sentences on a 1
  to 7-point Likert scale (1 the worst, 7 the best)
\item
  both parts: all conditions were crossed with two conditions:

  \begin{itemize}
  \tightlist
  \item
    neg-words
  \item
    strong NPIs
  \end{itemize}
\end{itemize}
\end{frame}

\begin{frame}
\begin{block}{Experiment: part 1 (example item)}
\protect\hypertarget{experiment-part-1-example-item}{}
\ex. \ag. V království nezůstal \{žádný/ani~jeden\} zloděj.\\
in kingdom neg-ramained neg-word/NPI thief\\
`No thief remained in the kingdom.' \bg. Král nechce, aby v království
zůstal \{žádný/ani jeden\} zloděj.\\
King neg-wants that in kingdom remained neg-word/NPI thief\\
`The king doesn't want any thief to remain in the kingdom.' \cg. Zloděj
ze souostroví Qwghlm je tak šikovný jako \{žádný/ani jeden\} zloděj.\\
thief from archipelago Qwghlm is so clever how neg-word/NPI thief\\
`The thief from the Qwghlm archipelago is as clever as no other thief.'

\begin{itemize}
\tightlist
\item
  first part: 3x2 design
\end{itemize}
\end{block}
\end{frame}

\begin{frame}
\begin{block}{Experiment: part 2}
\protect\hypertarget{experiment-part-2}{}
\begin{itemize}
\tightlist
\item
  in this part, the two classes of negative dependent expressions were
  tested against a manipulated context
\item
  the context was created to fix a scale (probability, noteworthiness,
  \ldots)
\item
  both neg-words and strong NPIs were tested with tops and bottoms of
  the contextual scale

  \begin{itemize}
  \tightlist
  \item
    2x2 design
  \item
    neg-words/strong NPIs vs.~top-of-the scale/bottom of the scale
  \end{itemize}
\end{itemize}
\end{block}
\end{frame}

\begin{frame}
\ex. Kontext: Šikovný trpaslík ze vsi najde v těchhle dolech za den 1, 2
někdy i 3 diamanty.\\
Context: A clever dwarf from the village will find 1, 2 or 3 diamonds in
these mines per day.\\
\ag. Jeden šikovný trpaslík ze vsi nenašel včera v dolech \{žádný/ani
1\} diamant.\\
one clever dwarf from village neg-found yesterday in mines neg-word/NPI
1 diamond\\
`One clever dwarf from the village didn't find even one diamond in the
mines yesterday.' \bg. Jeden šikovný trpaslík ze vsi nenašel včera v
dolech \{žádné/ani\} 3 diamanty.\\
one clever dwarf from village neg-found yesterday in mines neg-word/NPI
3 diamonds\\
`One clever dwarf from the village didn't find even three diamonds in
the mines yesterday.'

~
\end{frame}

\begin{frame}
\begin{block}{Results}
\protect\hypertarget{results}{}
\begin{figure}
\centering
\includegraphics{"error_bar.png"}
\caption{Graph of acceptance (+error bars)}
\end{figure}
\end{block}
\end{frame}

\begin{frame}
\begin{block}{Hierarchical models}
\protect\hypertarget{hierarchical-models}{}
(bottom of the scale probability, top in Appendix)

\begin{itemize}
\tightlist
\item
  mixed hierachical models with random effects for subjects and items
  (full structure: slope and intercept)
\item
  Cumulative Link Mixed Model: R package \texttt{ordinal}
  (\citet{r-ordinal})
\item
  multiple hierarchical regression with interaction (3x2 and 2x2)
\end{itemize}
\end{block}
\end{frame}

\begin{frame}
\begin{block}{Demographic factors}
\protect\hypertarget{demographic-factors}{}
\begin{itemize}
\tightlist
\item
  negative concord can vary depending on social factors (Montréal
  French: \citet{burnett2015variable} but also:
  \citet{burnett2018structural})

  \begin{itemize}
  \tightlist
  \item
    age, education level
  \end{itemize}
\item
  in the experiment, the subjects were asked for:

  \begin{itemize}
  \tightlist
  \item
    region
  \item
    age
  \item
    daily reading time (books, newspapers, \ldots)
  \end{itemize}
\end{itemize}

Summary of demographic factors:

\begin{itemize}
\tightlist
\item
  no interaction between neg-words or strong NPIs with either of the 3
  factors
\item
  no main effect
\item
  the variation effects discussed later are not social (the same
  results: after z-transformation of age)
\end{itemize}
\end{block}
\end{frame}

\begin{frame}
\begin{enumerate}
\item
  main effects: all conditions were degraded against the baseline
\item
  \textbf{interaction effects}:
\end{enumerate}

\begin{itemize}
\tightlist
\item
  the strong positive effect of neg-words by equatives
\item
  non-significant effect of neg-words by NegRaising (but see next exps
  and variation)
\item
  significantly strong negative effect of neg-words by probability
\end{itemize}

(the same results: Bayesian model -- Appendix)
\end{frame}

\begin{frame}[fragile]
\footnotesize

\begin{Shaded}
\begin{Highlighting}[]
\NormalTok{Cumulative Link Mixed Model fitted with the Laplace approximation}
\NormalTok{formula}\SpecialCharTok{:} \FunctionTok{as.factor}\NormalTok{(rating1) }\SpecialCharTok{\textasciitilde{}}\NormalTok{ condition }\SpecialCharTok{*}\NormalTok{ classNExpr }\SpecialCharTok{+}\NormalTok{ (}\DecValTok{1} \SpecialCharTok{+}\NormalTok{ condition }\SpecialCharTok{|}  
\NormalTok{  participant) }\SpecialCharTok{+}\NormalTok{ (}\DecValTok{1} \SpecialCharTok{+}\NormalTok{ condition }\SpecialCharTok{|}\NormalTok{ item)}
\NormalTok{  data}\SpecialCharTok{:}\NormalTok{    items\_with\_probability\_bott}

\NormalTok{Coefficients}\SpecialCharTok{:}
\NormalTok{                           Estimate Std. Error z value }\FunctionTok{Pr}\NormalTok{(}\SpecialCharTok{\textgreater{}}\ErrorTok{|}\NormalTok{z}\SpecialCharTok{|}\NormalTok{)    }
\NormalTok{conditionEq               }\SpecialCharTok{{-}}\FloatTok{5.31772}    \FloatTok{0.46257} \SpecialCharTok{{-}}\FloatTok{11.496}  \SpecialCharTok{\textless{}} \FloatTok{2e{-}16} \SpecialCharTok{**}\ErrorTok{*}
\NormalTok{conditionNR               }\SpecialCharTok{{-}}\FloatTok{5.62684}    \FloatTok{0.44530} \SpecialCharTok{{-}}\FloatTok{12.636}  \SpecialCharTok{\textless{}} \FloatTok{2e{-}16} \SpecialCharTok{**}\ErrorTok{*}
\NormalTok{conditionProb             }\SpecialCharTok{{-}}\FloatTok{5.62179}    \FloatTok{0.51548} \SpecialCharTok{{-}}\FloatTok{10.906}  \SpecialCharTok{\textless{}} \FloatTok{2e{-}16} \SpecialCharTok{**}\ErrorTok{*}
\NormalTok{classNExprZ               }\SpecialCharTok{{-}}\FloatTok{0.88195}    \FloatTok{0.26981}  \SpecialCharTok{{-}}\FloatTok{3.269}  \FloatTok{0.00108} \SpecialCharTok{**} 
\NormalTok{conditionEq}\SpecialCharTok{:}\NormalTok{classNExprZ    }\FloatTok{3.16921}    \FloatTok{0.32897}   \FloatTok{9.634}  \SpecialCharTok{\textless{}} \FloatTok{2e{-}16} \SpecialCharTok{**}\ErrorTok{*}
\NormalTok{conditionNR}\SpecialCharTok{:}\NormalTok{classNExprZ    }\FloatTok{0.06224}    \FloatTok{0.31883}   \FloatTok{0.195}  \FloatTok{0.84523}    
\NormalTok{conditionProb}\SpecialCharTok{:}\NormalTok{classNExprZ }\SpecialCharTok{{-}}\FloatTok{0.71610}    \FloatTok{0.33130}  \SpecialCharTok{{-}}\FloatTok{2.161}  \FloatTok{0.03066} \SpecialCharTok{*}  
\SpecialCharTok{{-}{-}{-}}
\NormalTok{Signif. codes}\SpecialCharTok{:}  \DecValTok{0} \StringTok{\textquotesingle{}***\textquotesingle{}} \FloatTok{0.001} \StringTok{\textquotesingle{}**\textquotesingle{}} \FloatTok{0.01} \StringTok{\textquotesingle{}*\textquotesingle{}} \FloatTok{0.05} \StringTok{\textquotesingle{}.\textquotesingle{}} \FloatTok{0.1} \StringTok{\textquotesingle{} \textquotesingle{}} \DecValTok{1}\StringTok{\textasciigrave{}\textasciigrave{}\textasciigrave{}}
\end{Highlighting}
\end{Shaded}
\end{frame}

\begin{frame}
\begin{block}{Summary}
\protect\hypertarget{summary}{}
\begin{enumerate}
\tightlist
\item
  neg-words are (unlike strong NPIs) accepted in the standard of
  equatives
\end{enumerate}

\begin{itemize}
\tightlist
\item
  unexplainable in the syntactic theory of neg-words
\item
  NPI unacceptability is surprising but probably results from
  cross-linguistic differences in equatives
\end{itemize}

\begin{enumerate}
\setcounter{enumi}{1}
\item
  NegRaising predicates are better licensors for strong NPIs (the effect
  was not significant in this experiment but see exp. evidence below)
\item
  in probability/scale manipulated contexts, strong NPIs are preferred
\end{enumerate}

\begin{itemize}
\tightlist
\item
  again problematic for the syntactic theory of neg-words
\end{itemize}

Intriguing correlations between conditions (per speaker).
\end{block}
\end{frame}

\begin{frame}
\begin{block}{Correlations}
\protect\hypertarget{correlations}{}
\begin{itemize}
\tightlist
\item
  all speakers agreed on their high acceptance of baseline
\item
  but some rated \emph{ani} high in equatives, those who accept it in
  NegRaising: strong NPI
\item
  similar observations in previous experiments: baselines universally
  accepted but divergent acceptability in non-baseline conditions
\item
  speakers who accept \emph{ani} in equatives treat it as neg-word
\item
  technically:

  \begin{itemize}
  \tightlist
  \item
    z-transformation of (by subject) acceptance of conditions
  \item
    checking the correlation of such z-transformed ratings
  \item
    Pearson's product-moment correlation: t = \(-5.93\), p-value
    \(< 0.001\)
  \end{itemize}
\item
  this is a continuation of \citet{docekaldotlacilsubber}: correlation
  between probability and NegRasing (for \emph{ani} but not for
  neg-words): see experiments below
\item
  but crucially, no correlations against the baseline: slide after the
  next slide
\end{itemize}
\end{block}
\end{frame}

\begin{frame}
\begin{figure}
\centering
\includegraphics{"correlations_ani.png"}
\caption{Correlations between NegRaising and Equatives}
\end{figure}
\end{frame}

\begin{frame}
\begin{figure}
\centering
\includegraphics{"equatives_baseline_corr.png"}
\caption{Correlations between Equatives and Baseline}
\end{figure}
\end{frame}

\begin{frame}
\begin{block}{Distribution and correlations summary}
\protect\hypertarget{distribution-and-correlations-summary}{}
\begin{longtable}[]{@{}
  >{\raggedright\arraybackslash}p{(\columnwidth - 12\tabcolsep) * \real{0.2708}}
  >{\raggedright\arraybackslash}p{(\columnwidth - 12\tabcolsep) * \real{0.1042}}
  >{\raggedright\arraybackslash}p{(\columnwidth - 12\tabcolsep) * \real{0.1250}}
  >{\raggedright\arraybackslash}p{(\columnwidth - 12\tabcolsep) * \real{0.0833}}
  >{\raggedright\arraybackslash}p{(\columnwidth - 12\tabcolsep) * \real{0.0833}}
  >{\raggedright\arraybackslash}p{(\columnwidth - 12\tabcolsep) * \real{0.1458}}
  >{\raggedright\arraybackslash}p{(\columnwidth - 12\tabcolsep) * \real{0.1875}}@{}}
\toprule()
\begin{minipage}[b]{\linewidth}\raggedright
\end{minipage} & \begin{minipage}[b]{\linewidth}\raggedright
Bas
\end{minipage} & \begin{minipage}[b]{\linewidth}\raggedright
Prob (unlik.)
\end{minipage} & \begin{minipage}[b]{\linewidth}\raggedright
Eq
\end{minipage} & \begin{minipage}[b]{\linewidth}\raggedright
NR
\end{minipage} & \begin{minipage}[b]{\linewidth}\raggedright
Fragm.
\end{minipage} & \begin{minipage}[b]{\linewidth}\raggedright
Without
\end{minipage} \\
\midrule()
\endhead
strong NPIs & \(\checkmark\) & \(\checkmark\) & * & \(\checkmark\)* & *
& \(\checkmark\) \\
neg-words & \(\checkmark\) & * & \(\checkmark\) & * & \(\checkmark\)* &
\(\checkmark\) \\
\bottomrule()
\end{longtable}

\begin{longtable}[]{@{}lllll@{}}
\toprule()
& Eq \ldots{} NR & Prob. \ldots{} NR & Fragm. \ldots{} NR & Eq \ldots{}
Bas \\
\midrule()
\endhead
strong NPIs & neg. corr. & neg. corr. & neg. corr. & * \\
neg-words & * & * & * & * \\
\bottomrule()
\end{longtable}
\end{block}
\end{frame}

\begin{frame}{Theoretical consequences}
\protect\hypertarget{theoretical-consequences}{}
\begin{block}{Assumptions: licensing of (strong) NPIs}
\protect\hypertarget{assumptions-licensing-of-strong-npis}{}
\begin{itemize}
\tightlist
\item
  general framework: mixture of \emph{even}-theory of NPIs licensing
  (\citealt{krifka1995semantics,lahiri1998focus,crnivc2014non} a.o.) and
  Gajewski's formalization of strong NPIs \cite{gajewski2011licensing}
\item
  licensing NPIs (after \citet{gajewski2011licensing}): strong NPIs are
  licensed in downward-entailing (DE) environments
\item
  DE both in Truth-Conditions (TC) but also in the non-at-issue meaning
\end{itemize}

\ex. An NPI is licensed in the environment \(\gamma\)\\
\([_\alpha exh [_\beta \ldots [_\gamma\) NPI \(] \ldots ]]\): \a. the
environment \(\gamma\) is DE in \(\beta\) \hfill weak NPIs \b. the
environment \(\gamma\) is DE in \(\alpha\) \hfill strong NPIs

~
\end{block}
\end{frame}

\begin{frame}
\begin{itemize}
\tightlist
\item
  the exhaustifier for strong NPIs as English \emph{even one}: covert
  \(even\)
\item
  the standard analysis for scalar strong NPIs \citet{crnivc2014against}
  and for scalar reading of focus particles \citet{panizza2020minimal}
\item
  overt but also covert \emph{even} has scalar \Next[a] and additive
  \Next[b] presupposition:
\item
  the presuppositions after \citet{panizza2020minimal} (the additive
  sometimes suspended):
\end{itemize}

\ex. \a. Even Pope\(_F\) danced. \b. Even one\(_F\) cat will make Pope
happy.

\ex. `Even \(\phi\)' presupposes: \a. that \(\phi\) is relatively
unlikely to be true among Alt(\(\phi\)); and \b. that there is
\(\psi \in\) Alt(\(\phi\)) that is not entailed by \(\phi\) and is true.

\footnotesize(for monotonic scales, likelihood translates into
entailment (after \citealt{crnic2011getting}))\normalsize
\end{frame}

\begin{frame}
\begin{block}{Baseline from the experiment}
\protect\hypertarget{baseline-from-the-experiment}{}
\begin{itemize}
\tightlist
\item
  \emph{ani} strong NPIs associate with covert \emph{even} (scope:
  propositional level)
\item
  it reqires DE both in TC and non-at-issue
\item
  plus the scalar presupposition of covert \emph{even} exhaustifier
\item
  the exhaustified focus alternatives: other cardinality predicates
  (after \citealt{lahiri1998focus,crnic2011getting} a.o.)
\item
  the entailment between numerals is reversed by negation:
  \(\neg (\llbracket\) one cat
  \(\rrbracket \ldots) \models \neg(\llbracket\)two
  cats\(\rrbracket \ldots)\)
\end{itemize}

\footnotesize

\ex. Ani one thief neg-remained in the kingdom.\\
\a. ~{[}\(_\alpha\) (\(even\)) {[}\(_\beta\)
\neg [$_\gamma$ ani one thief remained in the kingdom ]{]} {]} \a. TC
(in \(\beta\)) DE: \(\checkmark\) \b. non-at-issue (in \(\alpha\)) DE:
\(\checkmark\) \c. scalar presupposition of (even): \(\rightarrow\)
\(\neg\)(two thieves remained), \(\neg\)(three thieves remained),
\ldots: \(\checkmark\) \d. additive presupposition: \(\neg\)(two thieves
remained) \(\vee\) \(\neg\)(three thieves remained), \ldots:
\(\checkmark\) \z.

~

\normalsize
\end{block}
\end{frame}

\begin{frame}
\begin{block}{Other conditions from the experiment}
\protect\hypertarget{other-conditions-from-the-experiment}{}
\begin{block}{Likelihood}
\protect\hypertarget{likelihood}{}
\begin{itemize}
\tightlist
\item
  the explanation is the same as for the baseline: the scope (\(even\))
  \textgreater{} \(\neg\) \textgreater{} \ldots{} one \ldots{}
\item
  the general preference of strong NPIs over neg-words follows from the
  semantic theory of neg-words -- bellow
\end{itemize}
\end{block}

\begin{block}{Neg-Raising}
\protect\hypertarget{neg-raising}{}
\begin{itemize}
\tightlist
\item
  in many previous experiments (three at least): Neg-Raising was better
  accepted with strong NPIs (but the effect was never strong)

  \begin{itemize}
  \tightlist
  \item
    one possibility: the variation -- speakers who treat \emph{ani} as a
    neg-word blur the line
  \end{itemize}
\item
  standard theories of Neg-Raising: \cite{gajewski2007neg} or
  \cite{romoli2013scalar}

  \begin{itemize}
  \tightlist
  \item
    the scope of negation (via the excluded middle inference) on the
    embedded predicate
  \end{itemize}
\item
  at the embedded level: covert (\(even\)) \textgreater{} \(\neg\)
  \textgreater{} {[}\ldots{} one \ldots{]}
\item
  neg-words: the locality constraints -- see below
\end{itemize}
\end{block}
\end{block}
\end{frame}

\begin{frame}
\begin{block}{Neg-words}
\protect\hypertarget{neg-words}{}
\begin{itemize}
\tightlist
\item
  semantic/pragmatic theory of neg-words and negative concord
\item
  \citet{ovalle2004double} and modern reformulation in
  \citet{kuhn2022dynamics}
\item
  TC: indefinite description
\item
  non-at-issue: empty reference
\end{itemize}

\ex. \a.
\(\llbracket\)neg-word\(\rrbracket\)=\(\lambda P.\exists x[SORT(x) \wedge P(x)]\)
\hfill TC \b.
\(\llbracket\)neg-word\(\rrbracket\)=\(\neg \exists x[SORT(x) \wedge P(x)]\)
\hfill non-at-issue \a. after \citet{kuhn2022dynamics}:
\(\wedge \mathbf{0_x}\) \ldots postsupposition (highest scope) \z.

~
\end{block}
\end{frame}

\begin{frame}
\begin{block}{Locality, etc.}
\protect\hypertarget{locality-etc.}{}
\begin{itemize}
\tightlist
\item
  \citet{kuhn2022dynamics}: many improvements of
  \citet{ovalle2004double}
\item
  discourse referents (presupposed to be empty) are delimited by the
  previous context

  \begin{itemize}
  \tightlist
  \item
    more specific concerning the presupposition of emptiness
  \end{itemize}
\item
  neg-words are analyzed via split scope around licensor (prototypically
  negation)

  \begin{itemize}
  \tightlist
  \item
    the split scope is achieved via quantifier raising
  \item
    the locality constraints on neg-word licensing \(\approx\) QR in the
    particular language and construction
  \end{itemize}
\end{itemize}
\end{block}
\end{frame}

\begin{frame}
\begin{block}{Explaining the baseline}
\protect\hypertarget{explaining-the-baseline}{}
\ex. neg-word thief neg-remained in the kingdom. \a.
{[}\(\neg[\exists x[\mathbf{thief}(x) \wedge \mathbf{remained}(x)]\)
{]}{]} \(\wedge \mathbf{0_x}\)

\begin{itemize}
\tightlist
\item
  TC and the postsupposition are compatible
\item
  in positive sentences, the \(\mathbf{0_x}\) postsupposition leads to
  ungrammaticality:
\end{itemize}

\ex. neg-word thief remained in the kingdom. \a.
{[}\(\exists x[\mathbf{thief}(x) \wedge \mathbf{remained}(x)]\) {]}
\(\wedge \mathbf{0_x}\) \hfill \(\bot\)

\begin{itemize}
\tightlist
\item
  this also nicely explains the acceptability of neg-words with
  \emph{bez} `without' (no morphological negation)
\end{itemize}
\end{block}
\end{frame}

\begin{frame}
\begin{block}{Other conditions from the experiment}
\protect\hypertarget{other-conditions-from-the-experiment-1}{}
\begin{block}{Probability}
\protect\hypertarget{probability}{}
\begin{itemize}
\tightlist
\item
  both in top and bottom contexts, strong NPIs were preferred
\item
  the contexts were (nearly always) set up with positive inference
\item
  the positive inference goes against \(\mathbf{0_x}\) presupposition of
  neg-words

  \begin{itemize}
  \tightlist
  \item
    it can also explain the surprisingly high acceptability of strong
    NPIs even in top scalar contexts
  \item
    another factor: different scales (numerical in last experiment,
    ad-hoc in previous) \(\rightarrow\) future experimental work
  \end{itemize}
\end{itemize}
\end{block}
\end{block}
\end{frame}

\begin{frame}
\begin{block}{Neg-Raising}
\protect\hypertarget{neg-raising-1}{}
\begin{itemize}
\tightlist
\item
  previous experimental work: mostly evidence for decreased
  acceptability of neg-words (against strong NPIs) in NR
\item
  Kuhn's QR approach: explains the neg-words decreased acceptability
\item
  in the last experiment: the contrast is blurred
\item
  one possibility: to remove subjects treating \emph{ani} as a neg-word
  from the stats
\item
  unlike with equatives, the environment seems to be nearly as
  acceptable for neg-words as for strong NPIs
\end{itemize}
\end{block}
\end{frame}

\begin{frame}
\begin{block}{Equatives}
\protect\hypertarget{equatives}{}
\begin{itemize}
\tightlist
\item
  Slavic equatives are different from English equatives, and their
  morpho-syntax is very similar to correlatives

  \begin{itemize}
  \tightlist
  \item
    Slavic equatives are built on the correlative syntax
  \item
    and following \cite{pauline1995quantificational}: correlatives are
    bad licensors of NPIs
  \end{itemize}
\item
  another experiment in preparation: weak NPIs are penalized in Czech
  equatives (but acceptable in comparatives)

  \begin{itemize}
  \tightlist
  \item
    Slavic equatives are probably not even DE (as was observed for
    German: \citealt{krifka1992some,penka2016degree})
  \end{itemize}
\item
  neg-words are acceptable but verbal negation not (as in German:
  \citealt{penka2016degree})
\end{itemize}

\exg. Petr je tak chytrý jak nikdo jiný/*Marie ne.\\
Petr is so smart how neg-word else/Mary not\\
\hspace*{0.333em}

~
\end{block}
\end{frame}

\begin{frame}
\begin{block}{Equatives II}
\protect\hypertarget{equatives-ii}{}
\begin{itemize}
\tightlist
\item
  syntactic and semantic ingredients (pseudoCzech in \Next)
\item
  non-standard: \(max \rightarrow max_{inf}\) (otherwise \(max\) would
  lead to \(\bot\)): \cite{penka2016degree}
\end{itemize}

\ex. This thief is so clever how neg-word other thief.\\
\a. {[} so {[}so\(_1\) no other thief \(t_1\) clever {]}{]}\(_2\)
{[}This thief is \(t_2\) clever{]}\\
\b. \(\llbracket so\rrbracket\) \ldots{} picks up the degree denoted by
the standard clause\\
\b. \(\llbracket\) how\(_1\) neg-word other thief clever is
\(\rrbracket\)\\
\a. nobody other than the thief is \(d\)-clever \hfill neg-word
presupposition\\
\b. the thief is \(d\)-clever \hfill implicature of \emph{other}\\
\z.

\ex. \a. \(\llbracket\) as
\(\rrbracket = \lambda S\lambda C.max(C) \geq max(S)\)\\
\b.
\(S' \subseteq S: max(C) \geq max(S) \rightarrow max(C) \geq max(S')\)
\hfill English DE \textit{as}

~
\end{block}
\end{frame}

\begin{frame}
\begin{block}{Equatives III}
\protect\hypertarget{equatives-iii}{}
Motivation of the ingredients:

\begin{itemize}
\tightlist
\item
  \(max_{inf}\): the equative in Czech has exactly the same building
  blocks (\emph{tak} `so' \ldots{} \emph{jak} `how') as correlative
  constructions
\item
  \emph{other}: the anaphor similar to reciprocal anaphors

  \begin{itemize}
  \tightlist
  \item
    it identifies the dref
  \item
    it is also used in the exceptive phrases from which the
    presupposition comes: \emph{Nobody other than John neg-came}
    presupposes that John came (as the only exception)
  \end{itemize}
\item
  neg-word presupposition ranges over the dref picked up by the
  reciprocal
\end{itemize}
\end{block}
\end{frame}

\begin{frame}
\begin{block}{Summary 1}
\protect\hypertarget{summary-1}{}
\begin{itemize}
\tightlist
\item
  Czech neg-words and strong NPIs
\item
  existential TC core: \(\lambda P.\exists x[NP(x) \wedge P(x)]\)
\end{itemize}

\begin{longtable}[]{@{}lll@{}}
\toprule()
& TC & non-at-issue meaning \\
\midrule()
\endhead
neg-words & existential & \(\mathbf{0_x}\) \\
strong NPIs & existential & scalar presupposition \\
& & association with (even) \\
\bottomrule()
\end{longtable}
\end{block}
\end{frame}

\begin{frame}
\begin{block}{Summary 1}
\protect\hypertarget{summary-1-1}{}
\begin{itemize}
\tightlist
\item
  that explains (with some other more or less standard assumptions) the
  patterns of the experiment(s) plus:
\end{itemize}

\ex. How to explain the unpredicted acceptability of neg-words in
equatives (and NPIs unavailability)?

\ex. The non-standard \(max_{inf}\) accounts for the surprising
neg-words acceptability.\\
\a. decisive evidence for the semantic theory of neg-words\\
\b. non-monotonic environment: NPIs are predicted to be out

\begin{itemize}
\tightlist
\item
  neg-words in equatives: no standard theory of equatives with
  interpreted \(\neg\) ({[}uNeg{]}) in the standard
\end{itemize}
\end{block}
\end{frame}

\begin{frame}
\begin{itemize}
\tightlist
\item
  the answer to Question 2:
\end{itemize}

\ex. Question2: \a. How can we explain microvariation by grammatical
(semantic) factors? \b. Is part of the variation caused by social
factors? \z.

\ex. The speaker variation is explainable as shifting from the scalar to
the emptiness of the DR presupposition (in case of \emph{ani jeden}
`even one'). \a. Social factors don't seem to play a role in this shift.

\begin{itemize}
\tightlist
\item
  the experimental data support the semantic theory of neg-words: higher
  acceptability of strong NPIs in the probability manipulated contexts:
  unpredicted, many other environments (fragmentary answers preference
  for neg-words and also \emph{without} type of P)
\end{itemize}
\end{frame}

\begin{frame}
\begin{center}
\Huge Thanks!
\end{center}

\normalsize
\end{frame}

\begin{frame}
\begin{block}{Open questions}
\protect\hypertarget{open-questions}{}
\begin{itemize}
\tightlist
\item
  proper investigation of locality constraints

  \begin{itemize}
  \tightlist
  \item
    NegRaising: the concurrence sometimes vanishes (Maximize
    Presupposition of \citealt{heim1991articles}?)
  \end{itemize}
\item
  both scopes of covert \emph{even} in probability contexts (exp1) or
  just one (exp2 \& exp3), or the difference comes from different
  scales?
\item
  cross-linguistic variation in the neg-words locality: at least in some
  Romance languages, neg-words are licensed in \emph{before}-clauses and
  under \emph{doubt}-type of predicates

  \begin{itemize}
  \tightlist
  \item
    some suggestions in \cite{kuhn2022dynamics}
  \end{itemize}
\end{itemize}
\end{block}
\end{frame}

\begin{frame}{Appendix}
\protect\hypertarget{appendix}{}
\begin{block}{Histograms}
\protect\hypertarget{histograms}{}
\begin{figure}
\centering
\includegraphics{"histogram_faceted_prob_bott.png"}
\caption{Histogram: probabilities Bottom of the scale}
\end{figure}
\end{block}
\end{frame}

\begin{frame}
\begin{figure}
\centering
\includegraphics{"histogram_faceted_prob_top.png"}
\caption{Histogram: probabilities Top of the scale}
\end{figure}
\end{frame}

\begin{frame}
\begin{block}{Demographic factors II}
\protect\hypertarget{demographic-factors-ii}{}
\begin{enumerate}
\tightlist
\item
  region:
\end{enumerate}

\begin{itemize}
\tightlist
\item
  all regions of the Czech Republic aggregated to Bohemia vs.~Moravia:
\item
  67\% of subjects were from Bohemia, 33\% from Moravia
\item
  no significant main or interaction effect was found
\end{itemize}

\begin{enumerate}
\setcounter{enumi}{1}
\tightlist
\item
  age:
\end{enumerate}

\begin{itemize}
\tightlist
\item
  range: 19 to 71 years, mean: 25.6, median: 23
\item
  only significant interaction effect: younger people (under 27) rated
  probability condition slightly better (t-value: \(2.02\), p
  \(< 0.05\))
\end{itemize}
\end{block}
\end{frame}

\begin{frame}
\begin{block}{Demographic factors III}
\protect\hypertarget{demographic-factors-iii}{}
\begin{enumerate}
\setcounter{enumi}{2}
\tightlist
\item
  reading time
\end{enumerate}

\begin{itemize}
\tightlist
\item
  a proxy for education bias
\item
  reading time of books and other media: 0 to 10 hours
\item
  mean: 1.43, median: 1 hour
\item
  only one significant interaction: subjects with reading time
  \textgreater{} 1 hour rated NR-condition better (t-value \(2.05\), p
  \(< 0.05\))
\end{itemize}
\end{block}
\end{frame}

\begin{frame}
\begin{block}{More models}
\protect\hypertarget{more-models}{}
\begin{itemize}
\tightlist
\item
  Bayesian model for experiment 1: next slide
\item
  confidence intervals agree with p-values from the cumulative mixed
  model
\end{itemize}
\end{block}
\end{frame}

\begin{frame}
\begin{figure}
\centering
\includegraphics{"acc-results-complex.png"}
\caption{Bayesian model}
\end{figure}
\end{frame}

\begin{frame}[fragile]
\begin{itemize}
\tightlist
\item
  mixed linear model for the top of the scale (probability)
\end{itemize}

\footnotesize

\begin{Shaded}
\begin{Highlighting}[]
\NormalTok{Cumulative Link Mixed Model fitted with the Laplace approximation}

\NormalTok{formula}\SpecialCharTok{:} \FunctionTok{as.factor}\NormalTok{(rating1) }\SpecialCharTok{\textasciitilde{}}\NormalTok{ condition }\SpecialCharTok{*}\NormalTok{ classNExpr }\SpecialCharTok{+}\NormalTok{ (}\DecValTok{1} \SpecialCharTok{+}\NormalTok{ condition }\SpecialCharTok{|}  
\NormalTok{    participant) }\SpecialCharTok{+}\NormalTok{ (}\DecValTok{1} \SpecialCharTok{+}\NormalTok{ condition }\SpecialCharTok{|}\NormalTok{ item)}
\NormalTok{data}\SpecialCharTok{:}\NormalTok{    items\_with\_probability\_top}

\NormalTok{Coefficients}\SpecialCharTok{:}
\NormalTok{                          Estimate Std. Error z value }\FunctionTok{Pr}\NormalTok{(}\SpecialCharTok{\textgreater{}}\ErrorTok{|}\NormalTok{z}\SpecialCharTok{|}\NormalTok{)    }
\NormalTok{conditionEq               }\SpecialCharTok{{-}}\FloatTok{5.41517}    \FloatTok{0.46741} \SpecialCharTok{{-}}\FloatTok{11.585}  \SpecialCharTok{\textless{}} \FloatTok{2e{-}16} \SpecialCharTok{**}\ErrorTok{*}
\NormalTok{conditionNR               }\SpecialCharTok{{-}}\FloatTok{5.72051}    \FloatTok{0.44478} \SpecialCharTok{{-}}\FloatTok{12.861}  \SpecialCharTok{\textless{}} \FloatTok{2e{-}16} \SpecialCharTok{**}\ErrorTok{*}
\NormalTok{conditionProb             }\SpecialCharTok{{-}}\FloatTok{4.59856}    \FloatTok{0.56993}  \SpecialCharTok{{-}}\FloatTok{8.069}  \FloatTok{7.1e{-}16} \SpecialCharTok{**}\ErrorTok{*}
\NormalTok{classNExprZ               }\SpecialCharTok{{-}}\FloatTok{0.88066}    \FloatTok{0.27140}  \SpecialCharTok{{-}}\FloatTok{3.245} \FloatTok{0.001175} \SpecialCharTok{**} 
\NormalTok{conditionEq}\SpecialCharTok{:}\NormalTok{classNExprZ    }\FloatTok{3.21934}    \FloatTok{0.33077}   \FloatTok{9.733}  \SpecialCharTok{\textless{}} \FloatTok{2e{-}16} \SpecialCharTok{**}\ErrorTok{*}
\NormalTok{conditionNR}\SpecialCharTok{:}\NormalTok{classNExprZ    }\FloatTok{0.05194}    \FloatTok{0.32027}   \FloatTok{0.162} \FloatTok{0.871180}    
\NormalTok{conditionProb}\SpecialCharTok{:}\NormalTok{classNExprZ }\SpecialCharTok{{-}}\FloatTok{1.16830}    \FloatTok{0.32806}  \SpecialCharTok{{-}}\FloatTok{3.561} \FloatTok{0.000369} \SpecialCharTok{**}\ErrorTok{*}
\SpecialCharTok{{-}{-}{-}}
\NormalTok{Signif. codes}\SpecialCharTok{:}  \DecValTok{0}\NormalTok{ ‘}\SpecialCharTok{**}\ErrorTok{*}\NormalTok{’ }\FloatTok{0.001}\NormalTok{ ‘}\SpecialCharTok{**}\NormalTok{’ }\FloatTok{0.01}\NormalTok{ ‘}\SpecialCharTok{*}\NormalTok{’ }\FloatTok{0.05}\NormalTok{ ‘.’ }\FloatTok{0.1}\NormalTok{ ‘ ’ }\DecValTok{1}
\end{Highlighting}
\end{Shaded}

--\textgreater{}
\end{frame}

\begin{frame}
\end{frame}

\renewcommand\refname{References}
\begin{frame}[allowframebreaks]{References}
  \bibliographytrue
  \bibliography{bibliography.bib}
\end{frame}

\end{document}
